%!TEX root = ../surface_reconstruction.tex

\section{Ключевое пространство криптосистемы Мак-Элиса.}
В исследованиях криптосистем с открытым ключом возникает вопрос о
числе  открытых ключей, так как некоторые разные секретные ключи
могут порождать одинаковые открытые ключи. Если этих ключей
достаточно мало, то злоумышленник сможет эффективно строить по
открытому ключу некоторого легального абонента свой секретный
ключ. Что даст ему возможность читать все секретные сообщения,
приходящие в адрес абонента. Идеально, когда в криптосистеме любые
два различных секретных ключа порождают неравные друг другу
открытые ключи. Однако, часто в кодовых криптосистемах это не так.
В связи с чем возникает естественный вопрос, а сколько всего
открытых ключей.

Везде в этом параграфе предполагается, что в криптосистеме
Мак-Элиса в качестве порождающей матрицы кода $\mathcal C$ была
выбрана матрица $G$, а сам код имеет длину равную $n$, размерность
--- $k$ и кодовое расстояние --- $d$.  Легальный абонент в
качестве своего секретного ключа выбрал тройку $(H,G,\Gamma)$
(здесь $G$ включается в секретный ключ для удобства, на самом
деле, $G$ --- общедоступная матрица, выбираемая заранее), тогда
соответствующий открытый ключ будет равен произведению этих матриц
$H\cdot G\cdot \Gamma$. Введём отношение эквивалентности секретных
ключей следующим способом \Def{Два секретных ключа
$(H',G,\Gamma')$ и $(H'',G,\Gamma'')$ назовём
\emph{эквивалентными}, если и только если выполняется соотношение
$$
H'\cdot G\cdot\Gamma'=H''\cdot G\cdot\Gamma'',
$$
то есть порождаемые ими открытые ключи совпадают.}

Легко видеть, что данное отношение --- отношение эквивалентности.
Тем самым всё множество секретных ключей разбивается на классы
эквивалентности. В дальнейшем класс с представителем
$(H,G,\Gamma)$ будем обозначать как $[(H,G,\Gamma)]$

Важную роль в исследовании классов эквивалентности
$[(H,G,\Gamma)]$ играет группа автоморфизмов кода $\mathcal C$,
напомним ее определение.

\Def{\label{opr1}\emph{Группой автоморфизмов} кода $\mathcal C$
называется множество
$$
Aut(\mathcal C)=\{\Gamma\in S_n|\exists\; A\in GL_k(F_q):
G\Gamma=AG\},
$$
где $S_n$ --- симметрическая группа степени $n$, $GL_k(F_q)$ ---
группа всех невырожденных $k\times k$-матриц над полем $F_q$, а
перестановка $\Gamma\in S_n$ представляется перестановочной
$k\times k$-матрицей, которая действует на $G$ как соответствующая
перестановка столбцов. }

 С группой автоморфизмов неразрывно связано множество $\mathcal
A(\mathcal C)$ невырожденных $k\times k$-матриц, задающих
перестановки из группы автоморфизмов, то есть
$$
\mathcal A(\mathcal C)=\{A\in GL_k(F_q)|\exists\; \Gamma\in
Aut(G): AG=G\Gamma\}.
$$
\begin{proposition}
Пусть кодовое расстояние кода $\mathcal C^{\perp}$, дуального к
коду $\mathcal C$, строго больше двух. Тогда для любой матрицы
$A\in GL_k(F_q)$, принадлежащей множеству $\mathcal A(\mathcal
C)$, существует и единственная перестановка $\Gamma$ из группы
автоморфизмов $Aut(\mathcal C)$. И для любой перестановки
$\Gamma\in Aut(\mathcal C)$ существует и единственная матрица
$A\in\mathcal A(\mathcal C)$.
\end{proposition}
\begin{proof}
Пусть $A$ принадлежит $\mathcal A(\mathcal C)$, тогда найдётся
перестановка $\Gamma$ такая, что
$$
AG=G\Gamma.
$$
Очевидно, что $\Gamma\in Aut(\mathcal C)$. Докажем, что такая
$\Gamma$ единственна. Пусть существуют две перестановки $\Gamma'$
и $\Gamma''$ такие, что
$$
AG=G\Gamma'\;\;\;AG=G\Gamma''.
$$
Тогда
$$
G\Gamma'\Gamma''^{-1}=G.
$$
В силу того, что кодовое расстояние $\mathcal C^{\perp}$ строго
больше двух, то в матрице $G$ нет двух одинаковых столбцов.
Поэтому из последнего соотношения следует, что
$$
\Gamma'\Gamma''^{-1}=E.
$$
Полученное соотношение, означает, что $\Gamma'=\Gamma''$.

Докажем, что для $\Gamma\in Aut(\mathcal C)$ найдётся единственная
матрица $A\in \mathcal A(\mathcal C).$ В силу того, что $\Gamma\in
Aut(\mathcal C)$, то существует $A$, с которой
$$
AG=G\Gamma.
$$
Понятно, что $A\in\mathcal A(\mathcal C)$. Если найдутся две
матрицы $A'$ и $A''$ такие, что
$$
A'G=G\Gamma\;\;A''G=G\Gamma,
$$
то $A'G=A''G$. Ранг матрицы $G$ равен $k$, поэтому $A'=A''$.

Утверждение полностью доказано.
\end{proof}

Известно следующее утверждение, устанавливающее связь между
группой автоморфизмов кода $\mathcal C$ с порождающей матрицей $G$
и эквивалентными ключами криптосистемы Мак-Элиса.
\begin{proposition}\label{prop4}
Пусть в качестве порождающей матрицы кода $\mathcal C$ в
криптосистеме Мак-Элиса взята матрица $G$. Тогда существует
взаимно однозначное соответствие между множеством $Aut(\mathcal
C)$ и классом эквивалентности $[(H,G,\Gamma)]$ множества секретных
ключей криптосистемы Мак-Элиса.
\end{proposition}
\begin{proof}
Действительно, поставим секретному ключу
$(H',G,\Gamma')\in [(H,G,\Gamma)]$ в соответствие перестановку
$\Gamma'\Gamma^{-1}$. Докажем, что это перестановка принадлежит
группе автоморфизмов кода $\mathcal C$. В силу того, что
$(H',G,\Gamma')$ принадлежит классу $[(H,G,\Gamma)]$, то
выполняется соотношение
$$
HG\Gamma=H'G\Gamma'.
$$
Умножая правую и левую его части справа на матрицу $\Gamma^{-1}$, а
слева на $H'^{-1}$, получаем, что для перестановки
$\Gamma'\Gamma^{-1}$ справедливо соотношение
$$
H'^{-1}HG=G\Gamma'\Gamma^{-1}.
$$
Последнее и означает, что $\Gamma'\Gamma^{-1}$ принадлежит группе
автоморфизмов кода $\mathcal C$. Очевидно, данное  отображение
сюръективно. Покажем, что оно инъективно. Пусть существуют два
секретных ключа $(H',G,\Gamma')$ и  $(H'',G,\Gamma'')$ из класса
эквивалентности $[(H,G,\Gamma)]$, для которых
$\Gamma'\Gamma^{-1}=\Gamma''\Gamma^{-1}$. Тогда,
$\Gamma'=\Gamma''$. И, так как $H'G\Gamma'=H''G\Gamma''$, то равны
матрицы $H'$ и $H''$. Из всего сказанного следует, что ключи
$(H',G,\Gamma')$ и $(H'',G,\Gamma'')$ совпадают.

Утверждение полностью доказано.
\end{proof}

\begin{proposition}
Класс эквивалентности $[(H,G,\Gamma)]$ состоит из всех троек вида
$(H\cdot D_{\Gamma_A},G,\Gamma^{-1}_A\cdot \Gamma)$, где
$\Gamma_A\in Aut(\mathcal C)$, а $D_{\Gamma_A}G=G\Gamma_A$.
\end{proposition}
\begin{proof}
При доказательстве утверждения~\ref{prop4} было получено, что для
любого элемента $(H',G,\Gamma')\in[(H,G,\Gamma)]$ матрица
$H'^{-1}H$ принадлежит множеству $\mathcal A(\mathcal C)$, а
$\Gamma'\Gamma^{-1}$ --- соответствующая перестановка из группы
автоморфизмов кода $\mathcal C$, то есть $\Gamma'\Gamma^{-1}\in
Aut(\mathcal C)$. Если теперь положить $D_{\Gamma_A}=H'^{-1}H$ и
$\Gamma_A=\Gamma'\Gamma^{-1}$, то получим требуемое.
\end{proof}

Из этого утверждения немедленно получаем формулу для мощности
множества открытых ключей, которая совпадает с числом классов
эквивалентности секретных ключей.

\begin{proposition}\label{prop5}
Пусть $\mathcal E$ --- множество всех открытых ключей
криптосистемы Мак--Элиса. Тогда справедлива формула
$$
|\mathcal E|=\frac{n!h_k(q)}{|Aut(\mathcal C)|},
$$
здесь $h_k(q)=|GL_k(F_q)|=(q^k-1)(q^k-q)\ldots (q^k-q^{k-1})$ ---
число невырожденных матриц порядка $k$ над полем $F_q$.
\end{proposition}
\begin{proof}
Из утверждения~\ref{prop4} следует, что каждый класс
эквивалентности состоит из одного и того же числа элементов, а
именно из $|Aut(\mathcal C)|$ элементов, поэтому число классов, а
значит и число открытых ключей, будет равно отношению общего числа
секретных ключей к мощности класса эквивалентности. Всего
секретных ключей столько сколько пар $(H,\Gamma)$, где $H$
--- невырожденная $k\times k$-матрица, а $\Gamma$ ---
перестановочная $n\times n$-матрица. Число матриц $H$ равно
$h_k(q)$, а число всех перестановок --- $n!$, поэтому общее число
пар $(H,\Gamma)$ в точности равно $n!h_k(q)$. Учитывая всё
сказанное, получаем требуемую формулу.
\end{proof}
Заметим, что формула утверждения~\ref{prop5} даёт также общее
число  классов эквивалентности секретных ключей криптосистемы
Мак-Элиса.

Тем самым вопрос изучения классов эквивалентности секретных ключей
криптосистемы Мак-Элиса сводится к известной и трудной задаче
теории кодирования
--- описание группы автоморфизмов кода.

Одним из примеров кодов, для которых полностью вычислена группа
автоморфизмов, является двоичный код Рида--Маллера $RM(r,m)$, см.\
например~\cite{McWilliams}. Его группа автоморфизмов
$Aut(RM(r,m))$ представляет из себя полную аффинную группу
пространства $F^m_2$. Мощность этой группы равна $2^mh_m(2)$.
Поэтому справедливо утверждение.

\begin{proposition}
Общее число открытых ключей в криптосистеме Мак-Элиса, в которой в
качестве матрицы $G$ была выбрана порождающая матрица двоичного
кода Рида--Маллера, вычисляется по формуле
$$
|\mathcal E|=\frac{n!h_k}{2^mh_m},
$$
здесь $h_k=h_k(2)$ и $h_m=h_m(2)$.
\end{proposition}

Информацию о группах автоморфизмов других известных кодов можно
найти в~\cite{HandBook}.