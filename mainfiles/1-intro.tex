%!TEX root = ../surface_reconstruction.tex
\phantomsection
\section*{Введение} 
\addcontentsline{toc}{section}{Введение}
Проблема определения поверхности по набору точек активно изучается многие годы.






%Криптосистема Мак-Элиса "--- одна из старейших криптосистем с открытым ключом. Она была предложена в 1978
%Р.~Дж.~Мак-Элисом~\cite{MCEliece}. Данная криптосистема
%основывается на $\mathbf{\mathbb {NP}}$-трудной проблеме в теории
%кодирования. Основная идея её построения  состоит в маскировке
%некоторого кода, имеющего эффективные алгоритмы декодирования, под
%код, не обладающий видимой алгебраической и комбинаторной
%структурой, такие коды принято называть кодами общего положения.
%Эта криптосистема обладает одним важным преимуществом "--- высокой
%скоростью зашифрования и расшифрования. Однако, у неё имеется
%серьёзный недостаток "--- относительно низкая скорость передачи
%($R$). Обычно у кодовых криптосистем $R<1$, тогда как у
%криптосистемы RSA скорость в точности равна $1$.
%
%В этой работе рассматривается обобщение криптосистемы
%Мак-Элиса, предложенное в 1994 коду В.М.
%Сидельниковым~\cite{Sidelnikov1}. В этой работе модификация,
%предложенная В.~М.~Сидельниковым, называется криптосистемой\\
%Мак-Элиса--Сидельникова. Криптосистема Мак-Элиса--Сидельникова
%строится на основе $u$-кратного использования кодов Рида--Маллера
%$RM(r,m)$. Она имеет высокую криптографическую стойкость, скорость
%передачи близкую к $1$ и сравнительно невысокую сложность
%шифрования секретных сообщений и расшифрования криптограмм этих
%сообщений.
%
%В работе исследуются вопросы, связанные с пространством
%эквивалентных секретных ключей, то есть секретных ключей,
%порождающих одинаковые открытые ключи, новой криптосистемы. Опишем
%краткое содержание разделов работы.
%
%В \S~1 даётся определение криптосистемы Мак-Элиса, описываются её
%секретный и открытый ключи. Приводятся алгоритмы зашифрования и
%расшифрования.
%
%В \S~2 изучается ключевое пространство криптосистемы Мак-Элиса.
%Устанавливается связь классов эквивалентностей секретных ключей с
%группой автоморфизмов линейного кода, лежащего в основе этой
%криптосистемы.
%
%В \S~3 описывается криптосистема Мак-Элиса--Сидельникова:
%секретный и открытый ключи, алгоритмы зашифрования и
%расшифрования.
%
%\S~4 посвящён ключевому пространству новой криптосистемы. В нём
%вводятся множества, необходимые для описания классов
%эквивалентности секретных ключей. Получаются нижние и верхние
%оценки на мощности  введённых множеств и на число открытых ключей
%криптосистемы Мак-Элиса--Сидельникова.
%
%В \S~5 изучается криптосистема Мак-Элиса--Сидельникова в случае
%двух блоков ($u=2$).
%
%В настоящей работе получаются нижние оценки на мощность множества
%открытых ключей криптосистемы
%Мак-Элиса--Сидельникова(теорема~\ref{t3}) при использовании
%произвольного числа блоков $u$. Для кодов Рида--Маллера с
%$u$-кратным повторением строится множество, которое, в некотором
%смысле, является аналогом группы автоморфизмов обычного кода
%Рида--Маллера, и устанавливается связь этого множества с классами
%эквивалентности секретных ключей.
%
%Для случая двух блоков ($u=2$) полностью описывается указанное
%множество при использовании кодов Рида--Маллера $RM(r,m)$
%$(r\leqslant 2,r<m)$ и матриц определённого вида
%(теоремы~\ref{theorem1},~\ref{theorem2}). Тем самым при $u=2,
%r\geqslant 2, r<m$ описываются все классы эквивалентности
%секретных ключей с представителями особого вида и вычисляются их
%мощности. Для некоторых классов эквивалентности секретных ключей
%приводятся нижние оценки на их мощность(теоремы~\ref{theorem1}
%и~\ref{theorem2}).
