%!TEX root = ../surface_reconstruction.tex

\section*{Апробация} 
\addcontentsline{toc}{section}{Апробация}
М.И. Хабибулин. Исследование эффективности метода движущихся наименьших квадратов при реконструкции трёхмерной поверхности на суперкомпьютере.  Конференция "Ломоносов": труды международной научной конференции "Ломоносов". 10-21 апреля 2023 г., C. 40-43, Москва.

\section*{Основные результаты}
\addcontentsline{toc}{section}{Основные результаты}
\begin{itemize}
    \item Проанализированы наиболее популярные методы представления и реконструкции поверхности.
    \item Реализованы следующие варианты параллельного алгоритма реконструкции поверхности:
для систем с общей памятью (с использованием OpenMP), c распределённой памятью (с использованием MPI),
а также гибридный вариант (MPI + OpenMP)
    \item Исследована эффективность, разработанного алгоритма, а также проведены тесты реконструкции реальных поверхностей.
    \item Вычислительные эксперименты показали эффективность разработанных реализаций
\end{itemize}

\section*{Выводы и заключение}
\addcontentsline{toc}{section}{Выводы и заключение}
В работе рассмотрены методы реконструкции представляющие поверхность различными способами. Триангуляция Делоне представляет поверхность полигональной сеткой. Метод радиально базисных функций представляет поверхность набором неявных функций.
Оператор локально-оптимальной проекции и метод движущихся наименьших квадратов являются промежуточным вариантом и представляют поверхность набором точек. Оба алгоритма также могут повышать плотность точек до разрешения экрана и использоваться для последующего рендеринга. Довольно часто алгоритмы представляющие поверхность набором точек, используются для получения промежуточного результата и последующего применения алгоритмов представляющих поверхность полигональной сеткой или набором неявных функций. На сегодняшний день представление поверхности полигональной сеткой является наиболее используемой. Несмотря на то, что авторы ~\cite{CARR} ~\cite{Turk} утверждают, что представление поверхности неявными функциями имеет широкое применение, на сегодняшний день, представление поверхности полигональной сеткой вытеснило этот подход. Этому свидетельствует отсутствие возможности рендеринга сложных поверхностей представленных неявными функциями на наиболее популярном программном обеспечение для рендеринга (Unity, Blender и др.).

В работе сформулирован и реализован параллельный алгоритм реконструкции поверхности с использованием программного интерфейса Message Passing Interface для передачи сообщений, завязанный на алгоритме движущихся наименьших квадратов. Алгоритм предусматривает равномерное распределение сегментов облака точек по процессам с последующей пересылкой границ разбиения по топологии кольцо. Все последующие вычисления выполняются локально. Будучи избавленным от последующих пересылок данных, он обеспечивает максимальное ускорение.

Алгоритм также обладает большим ресурсом параллелизма ввиду нелинейного уменьшения количества операций на некоторых этапах алгоритма  с линейным увеличением количества процессов (см. табл. ~\ref{table:complexity}). Это касается этапов построение k-d-дерева и нахождения соседей для всех точек сегмента распределенного на процесс.

По результатам исследований алгоритм MLS хорошо справляется с зашумленными данными, но для достижения оптимальных результатов нужно тщательно подбирать параметр радиуса алгоритма. Достоинством алгоритма также стоит отметить всего один свободный параметр R (радиус поиска соседних точек) против трех у триангуляции Делоне. В качестве направления дальнейшего развития алгоритма интересно формулирование руководства выбора оптимального параметра алгоритма, автоматизация этого процесса.

