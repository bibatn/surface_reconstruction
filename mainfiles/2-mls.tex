%!TEX root = ../surface_reconstruction.tex


\section{Описание последовательно алгоритма}
Здесь будет описание последовательного алгоритма  

\includegraphics[scale=0.4]{0.png}


\section{Существующие алгоритмы}
\subsection{Оператор локально-оптимальной проекции(LOP)}
Происхождением метода является алгоритм Вайсфельда для решения задачи Ферма-Вебера о расположении точек, также известный как многомерная медиана L1. Это статистический инструмент, который традиционно применяется во всем мире для многомерных непараметрических точечных выборок, чтобы получить хороший представитель для большого количества выборок при наличии шума и выбросов. Проблема была впервые известна как проблема оптимального местоположения Вебера [1909]. Задача состояла в том, чтобы найти оптимальное место для промплощадки, минимизирующее стоимость доступа. В статистике проблема известна как медиана L1 [Brown 1983; Small 1990].

Задача Ферма-Вебера (глобальная) о расположении точек рассматривается как пространственная медиана, поскольку, будучи ограничена одномерным случаем, она совпадает с одномерной медианой и наследует некоторые ее свойства в многомерной постановке.

Реконструкция с помощью оператора проекции имеет важное достоинство: она определяет непротиворечивую геометрию на основе точек данных и предоставляет конструктивные средства для повышения ее дискретизации. 
Оператор локально-оптимальной проекции без параметризации использует более примитивный механизм проецирования, но поскольку он не основан на локальной 2D-параметризации, он более надежен и хорошо работает в сложных сценариях. Кроме того, если точки данных взяты локально с гладкой поверхности, оператор обеспечивает аппроксимацию второго порядка, что приводит к правдоподобной аппроксимации выбранной поверхности.

Оператор LOP имеет две непосредственные функции: во-первых, его можно использовать в качестве этапа предварительной обработки для любого другого метода реконструкции более высокого порядка (например, RBF). LOP можно применять к необработанным отсканированным данным для создания чистого набора данных, в качестве средства эффективного уменьшения шума и выбросов, а также для упрощения определения ориентации и топологии локальной поверхности. Во-вторых, его можно использовать для уточнения данного набора данных.

Для множества точек данных $P = \{p_j\} _j_\in_J \subset \mathbf R^{3}$, LOP проецирует произвольное множество точек $X^{(0)} = \{x_i^{(0)} \} _i_\in_I \subset \mathbf R^{3}$ 


\section{Параллелизм}

Здесь будет параллельный алгоритм на псевдокоде




Параллельный алгоритм

\includegraphics[scale=0.5]{1.png}

\section{Результаты экспериментов}

Время работы на n процессах n = 1...8


\includegraphics[scale=0.4]{T(P).png}

Ускорение:

\includegraphics[scale=0.4]{S(P).png}

Эффективность:

\includegraphics[scale=0.4]{E(P).png}

